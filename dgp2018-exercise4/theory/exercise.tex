\documentclass{scrartcl}
\usepackage[utf8]{inputenc}
\usepackage[english]{babel}
\usepackage{caption}
\usepackage{subcaption}
\usepackage{listings}
\usepackage{pdfpages}
\usepackage{amsmath,amssymb}
\usepackage{siunitx}
\usepackage{hyperref}
\usepackage{mhchem}
\usepackage[section]{placeins}
\usepackage[activate, protrusion=true, expansion=true]{microtype}
\usepackage[left=2.5cm, right=2.5cm, bottom=2.5cm, top=2.5cm]{geometry}
\usepackage{libertine}
\usepackage{longtable}

\newcommand{\qed}{\hfill $\blacksquare$}

\lstset{frame=single,keepspaces=true,captionpos=b}

\title{Digital 3D Geometry Processing}
\subtitle{Exercise 4}
\author{\textsc{Jannik Reichert} \and \textsc{Arne Sachtler} \and \textsc{Niklas Schmitz}}
\date{\today}

\begin{document}
\maketitle

\section{Theory Exercise}

\subsection{Curvature of Curves}

\begin{itemize}
	\item 1-d : 
	\item 2-a : 
	\item 3-c : 
	\item 4-b : 
\end{itemize}

\subsection{Surfaces Area}
% King Archimedes wants to renovate his palace. The most striking structure is a spherical
% half-dome of 20m in diameter that covers the great hall. The king wants to cover this
% dome in a layer of pure gold. He has decided to split the work into two parts, each one
% covering a vertical slice of the dome of the same height (see Figure 2). For each part he
% hires different people and gives them 700kg of gold. The task is to cover the surface of
% one vertical slice with a layer of gold of 0.1mm thickness. The amount of gold that is left
% over is the salary for doing the job. Which slice should you pick if you want to make the
% most profit? Explain your answer in TheoryExercise.pdf.
% How does your answer change when you have n slices instead of just two?

Preferable is the part, on which we have to use less gold to cover it.
We therefore calculate the surfaces of Parts A and B.
As we at first only care about which area is smaller, we can calculate it on the 
unit (half-)sphere.

We parametrize it as in lecture 4:
\begin{equation}
	\mathbf{x}(u,v) = 
	\begin{pmatrix}
		cos\,u\,sin\,v\\
		sin\,u\,sin\,v\\
		cos\,v
	\end{pmatrix}
	,\quad
	(u,v)\in[0,2\pi)\times [0,\frac{\pi}{2})
\end{equation}

A handy thing for areas on surfaces is the First Fundemental Form, which turns out to be
\begin{equation}
	\mathbf{I} =
	\begin{pmatrix}
		E & F\\
		F & G
	\end{pmatrix}
	=
	\begin{pmatrix}
		{\mathbf{x}_{,u}}^\top{\mathbf{x}_{,u}} & {\mathbf{x}_{,u}}^\top{\mathbf{x}_{,v}}\\
		{\mathbf{x}_{,u}}^\top{\mathbf{x}_{,v}} & {\mathbf{x}_{,v}}^\top{\mathbf{x}_{,v}}
	\end{pmatrix}
	=
	\begin{pmatrix}
		sin^2\,v & 0\\
		0 & 1
	\end{pmatrix}
\end{equation}

Now the area for (normalized) Part A can be calculated easily
\begin{eqnarray}
	\int_0^{\frac{\pi}{4}}\int_0^{2\pi} sin(v) \, du \, dv
	=
	\int_0^{\frac{\pi}{4}}[u\,sin\,v]_0^{2\pi} \, dv
	&=&
	2\pi\int_0^{\frac{\pi}{4}}sin\,v\,dv \\
	=
	2\pi(-\sqrt{2}+1)
	&=&
	(2-\sqrt{2})\pi
\end{eqnarray}
As a half circle has the total area $2\pi$, Part B must have area 
$\sqrt{2}\pi=2\pi-((2-\sqrt{2})\pi)$, which makes it larger than Part A.
Therefore Part A is preferable.

	
\end{document}
